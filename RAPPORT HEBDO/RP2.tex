\documentclass[a4paper,12pt]{report}
\usepackage[utf8]{inputenc}
\usepackage[T1]{fontenc}
\usepackage[french]{babel} 
\usepackage{xcolor,graphicx}
\usepackage{enumitem}
\usepackage{hyperref}
\usepackage[top=1cm, bottom=2cm, left=2cm, right=1cm]{geometry}
\graphicspath{{image/}}
\usepackage{graphicx}
\usepackage{array}
\linespread{1.5}


\begin{document}

\underline{\textbf{LE MICROCONTRÔLEUR}}\\[0.2cm]
Comme le dispositif doit être \underline{\textbf{portable}} et le capteur choisi n'ayant que peu de broche (le capteur a besoin des lignes séries SDA et SCL par contre), nos critères les plus importants seront : \textbf{la dimension}, \textbf{le nombre de broche (le moins de broche possible)}, \textbf{le mode de consommation (low, ultra-low)}.\\[1.5cm]


\underline{\textbf{TABLEAU COMPARATIFS DES µCONTRÔLEURS}}

\begin{table}[h!]
\centering
\begin{tabular}{|>{\raggedright\arraybackslash}m{4cm}|>{\raggedright\arraybackslash}m{3cm}|>{\raggedright\arraybackslash}m{3cm}|>{\raggedright\arraybackslash}m{3cm}|}\hline
\textbf{Critère} & \textbf{PIC16F887} & \textbf{Arduino Nano} \\
\hline
\textbf{Dimensions Physiques} & 4.44 x 4.44 x 0.80 cm & 4.5 x 1.8 x 1.5 cm \\
\hline
\textbf{Nombre de Broches} & 40 broches (PDIP) & 30 broches \\
\hline
\textbf{Modes de Consommation} &
Mode actif : typiquement 1.8 mA à 5V, 4 MHz\
Mode veille : typiquement 1.0 µA à 5V &
Mode actif : environ 19 mA à 5V \
Mode veille : environ 1 mA \\
\hline
\textbf{Horloge} & Jusqu'à 20 MHz (avec oscillateur externe) & 16 MHz \\
\hline
\textbf{Prix} & 15.000Ar & 50.000Ar 
\\
\hline
\end{tabular}
\end{table}


\underline{\textbf{INCONVENIENTS}}
\begin{itemize}
\setlength{\itemindent}{2cm}
\item \underline{\textit{\textbf{Pour le PIC16F887:}}}\\
Le problème pour le PIC est que la librairie officielle pour le capteur est inexistant voire introuvable, ce qui oblige à coder soi-même sa propre librairie, ou à copier les bouts de codes trouver sur internet.\\  

\pagenumbering{gobble}

\setlength{\itemindent}{2cm}
\item \underline{\textit{\textbf{Pour l'arduino NANO:}}}\\
Il n'est pas adapté pour les dispositifs embarqués mais plutôt pour les bancs d'essai et les prototypages.
\end{itemize}








\end{document}