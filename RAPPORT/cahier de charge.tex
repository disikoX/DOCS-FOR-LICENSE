
\documentclass[a4paper,12pt]{report}

\usepackage[utf8]{inputenc}
\usepackage[T1]{fontenc}
\usepackage[french]{babel} 
\usepackage{xcolor,graphicx}
\usepackage{enumitem}
\usepackage{lipsum}
\usepackage{pifont}
\usepackage{pdfpages}
\graphicspath{{image/}}
\usepackage[top=0.6in,bottom=0.6in,right=1in,left=1in]{geometry}
\usepackage{hyperref}
\hypersetup{
    colorlinks=true,
    linkcolor=dark,
    filecolor=magenta,      
    urlcolor=blue,
    }

\definecolor{blue}{RGB}{31,56,100}




\begin{document}



\begin{center}
	\begin{minipage}{2.5cm}
	\begin{center}
	\hspace{-3cm}
		\includegraphics[height=3cm]{ESP.png}	
	\end{center}
\end{minipage}\hfill
\begin{minipage}{13cm}
	\begin{center}
    \textbf{Ecole Supérieure Polytechnique Antsiranana}
    \center
    
    \textbf{B.P. O 201 - ANTSIRANANA-MADAGASCAR}\\[2mm]
    \hspace{-1cm}
    \textbf{Tél : +261(0)32 76 395 40 Courriel\textcolor{blue}{: mentionsticespa@gmail.com}}
 	\noindent\rule{\linewidth}{0.5pt}
     
	\end{center}
\end{minipage}\hfill \\[5mm]


\center
\textit{"Maîtriser aujourd'hui la technologie de demain"}\\ [0.5cm]
\center
\textbf{Mention Génie Electrcique et Technologique}\\[5mm]
\center


\center
\textbf{Projet de fin de semestre en L3 - A.U : 2023 - 2024 (1 étudiant)}\\[0.8cm]
\center 

\center
\textbf{BAROMETRE NUMERIQUE}\\[5mm]
\center

\hspace{-15cm}
\textbf{Objectif} \\[1mm]
\begin{flushleft}
Réalisation d’un dispositif capable de mesurer la pression. \\ [3mm]
\end{flushleft}

\hspace{-15cm}
\textbf{Principe} \\[5mm]
\begin{flushleft}
Ce dispositif affiche en alternance la pression atmosphérique en
millimètre de mercure et en hecto Pascal.
La réalisation emploie une carte à microcontrôleur (entre autres ARDUINO) \\ [3mm]
\end{flushleft}

\hspace{-13cm}
\textbf{Travaux demandés} \\[5mm]

\begin{itemize}
  \item[\ding{118}] Etude bibliographique sur le dispositif \\ [5mm]
   \item[\ding{118}]Proposer un schéma bloc et des schémas détaillés des éléments constitutifs du
système \\[5mm]
    \item[\ding{118}]Présenter un Schéma d’ensemble \\[5mm]
    \item[\ding{118}]Dimensionnement\\[5mm]
     \item[\ding{118}] Réalisation\\[5mm]
     \item[\ding{118}] Test et essais \\[5mm]
\end{itemize}


\renewcommand*{\FrenchLabelItem}{$\bullet$} %bulletpoint%

\hspace{-14cm}
\textbf{Lieu de travail}\\[5mm]
\begin{flushleft}
  \textbf{Laboratoire électronique general} \\ [5mm]
	
	 
\end{flushleft}

\hspace{-14.8cm}
\textbf{Encadreurs}\\[5mm]
\begin{flushleft}
	Mme FINOMANA Lydia \\[5mm]
	Mme TINA Marie Estella
\end{flushleft}

 %enlève les page%


\end{center}





\end{document}