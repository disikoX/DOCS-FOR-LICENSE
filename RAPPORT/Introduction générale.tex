\chapter*{Introduction générale}
\addcontentsline{toc}{chapter}{Introduction générale}
 
	Lorsque l'on veut obtenir des informations météorologiques, ou comprendre pourquoi à un niveau d'altitude donné l'air se raréfie, tout cela nous ramène à la pression.

Celle-ci est étudiée dans des domaines comme la thermodynamique, la mécanique des fluides, la navigation aérienne. De ce fait il est crucial d'améliorer son domaine de mesure.\\

Pendant des siècles la mesure de la pression fait d'énormes avancées mais la précision des résultats étaient pour la plupart aberrant. Mais avec les avancées en électronique, puis l'arrivée de capteurs performant, l'amélioration des instruments de mesures tels l'anémomètre, les altimètres, et enfin le baromètre, ont permis de plus grande plage mesure ainsi qu'une précision plus accrue.\\

Ce travail intitulé "Réalisation et conception d'un baromètre numérique" rentre dans le cadre d'un dispositif à but météorologique seulement. Il en découle, que le dispositif en question sera utilisé pour la collecte de données météo en temps réel.

Ainsi, pour mieux appréhender et analyser les détails de la réalisation et la conception du dispositif, nous avons réparti le travail en 3 parties. La première, qui concerne les généralités sur le baromètre, la pression atmosphériques, et quelques méthodes de mesure de celle-ci. La seconde, concernera l'étude et la conception du baromètre numérique. Finalement, la dernière partie portera sur la réalisation et les essais effectué sur celle-ci.     
    
\pagenumbering{arabic}
\setcounter{page}{1}       





